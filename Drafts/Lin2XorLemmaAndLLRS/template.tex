%%%%%%%%%%%%%%%%%%%%%%%%%%%%%%%%%%%%%%%%%
% Mathematical Cryptology Journal Article
% LaTeX Template
% Version 1.0 (19/10/2019)
%
% This template was created by:
% Vel (enquiries@latextypesetting.com)
% LaTeXTypesetting.com
%
% To compile the bibliography, first specify your BibTeX-formatted bibliography
% file in the \addbibresource{} command in this file. Then, compile the 
% bibliography using biber, either by navigating to the template directory and 
% entering 'biber template' in the command line, or by compiling from your LaTeX 
% editor after setting that you wish to use biber for compiling your bibliography.
%
%%%%%%%%%%%%%%%%%%%%%%%%%%%%%%%%%%%%%%%%%

%----------------------------------------------------------------------------------------
%	PACKAGES AND OTHER DOCUMENT CONFIGURATIONS
%----------------------------------------------------------------------------------------

\documentclass{mathcryptology} % Use the twocolumn option for typesetting article text in two columns

\addbibresource{sample.bib} % BibTeX bibliography file containing the article's references

%----------------------------------------------------------------------------------------
%	ARTICLE INFORMATION (TO BE SET BY EDITORS)
%----------------------------------------------------------------------------------------

\volume{1}
\issue{1}

\startpage{1}

\receiveddate{{\today}}
\revisiondate{{\today}}
\accepteddate{{\today}}

%----------------------------------------------------------------------------------------
%	ARTICLE INFORMATION (TO BE SET BY AUTHORS)
%----------------------------------------------------------------------------------------

\articletitle{Article Title}

% Authors with their affiliation numbers; affiliations are specified with: \aff{<numbers or symbols separated by commas>}
% Use a star (*) to specify the corresponding author
\articleauthors{Jennifer Lee\aff{1,*}, James Smith\aff{1,2}}

% Affiliation locations using the same numbers specified in \articleauthors above
\articleaffiliations{
	\aff{1}Department of Physics, Faculty of Science, University of Springfield\\
	\aff{2}Department of Statistics, Faculty of Science, Another University
}

\correspondingauthoremail{email@domain.com} % Email address of the corresponding author

\citationauthors{Lee, J. \& Smith J.} % Shorthand author list for the headers

\keywords{Keyword 1, Keyword 2, Keyword 3} % Keywords separated by commas

\MSClassification{94A60, 11Y40} % 2010 Mathematics Subject Classifications separated by commas

\abstract{
	Lorem ipsum dolor sit amet, consectetur adipiscing elit. Praesent porttitor arcu luctus, imperdiet urna iaculis, mattis eros. Pellentesque iaculis odio vel nisl ullamcorper, nec faucibus ipsum molestie. Sed dictum nisl non aliquet porttitor. Etiam vulputate arcu dignissim, finibus sem et, viverra nisl. Aenean luctus congue massa, ut laoreet metus ornare in. Nunc fermentum nisi imperdiet lectus tincidunt vestibulum at ac elit. Nulla mattis nisl eu malesuada suscipit. Aliquam arcu turpis, ultrices sed luctus ac, vehicula id metus. Morbi eu feugiat velit, et tempus augue. Proin ac mattis tortor. Donec tincidunt, ante rhoncus luctus semper, arcu lorem lobortis justo, nec convallis ante quam quis lectus. Aenean tincidunt sodales massa, et hendrerit tellus mattis ac. Sed non pretium nibh. Donec cursus maximus luctus. Vivamus lobortis eros et massa porta porttitor.
}

%----------------------------------------------------------------------------------------

\begin{document}

\begin{NoHyper}
\articleinformation % Output the article information section populated from the article information commands above
\end{NoHyper}

%----------------------------------------------------------------------------------------
%	ARTICLE BODY
%----------------------------------------------------------------------------------------

\section{Section}

Lorem ipsum dolor sit amet, consectetur adipiscing elit. Aliquam auctor mi risus, quis tempor libero hendrerit at. Duis hendrerit placerat quam et semper. Nam ultricies metus vehicula arcu viverra, vel ullamcorper justo elementum. Pellentesque vel mi ac lectus cursus posuere et nec ex. Fusce quis mauris egestas lacus commodo venenatis. Ut at arcu lectus. Donec et urna nunc. Morbi eu nisl cursus sapien eleifend tincidunt quis quis est. Donec ut orci ex. Praesent ligula enim, ullamcorper non lorem a, ultrices volutpat dolor. Nullam at imperdiet urna. Pellentesque nec velit eget est euismod pretium.

Donec in elit ac ante vestibulum rhoncus. Pellentesque congue ligula tortor, aliquet malesuada nulla tristique vitae. Aliquam mi sem, varius eu pellentesque et, tristique nec quam. Vestibulum pellentesque in dui et venenatis. Sed malesuada elit pellentesque sapien aliquet porta. In at facilisis diam. Duis id ante tellus. Aenean sit amet sem vel nisi finibus maximus et vitae nunc. Cras sit amet velit nec urna scelerisque porta. Pellentesque turpis ligula, aliquam eu quam eget, auctor fermentum risus.

Nulla facilisi. In hac habitasse platea dictumst. Proin non magna condimentum, porttitor leo sit amet, faucibus tellus. Integer tristique nulla vitae posuere lacinia. In venenatis et sapien consequat scelerisque. Nunc volutpat ipsum condimentum velit lacinia, vel feugiat purus sodales. Morbi rhoncus quam eu congue vestibulum. Sed tempor sed sapien a tristique. Curabitur efficitur justo aliquam mi rhoncus, ac elementum ex tincidunt. Ut condimentum, tortor eget faucibus volutpat, mauris libero vestibulum quam, ac pellentesque augue leo vel.

\subsection{Subsection}

In diam libero, vulputate quis accumsan non, auctor in ipsum. Praesent cursus velit eget lacus sodales porta. Proin quis risus ut velit euismod scelerisque ut sed neque. Cras sagittis, dolor ac ullamcorper auctor, tortor dui facilisis diam, at sagittis nisi ipsum a neque. Nullam vel mattis nisi. Ut interdum ut diam at ornare. Nulla ultrices elit justo, vitae tristique massa vulputate sit amet.

Etiam congue sem id orci faucibus, et iaculis ligula tincidunt. Aenean consectetur venenatis posuere. Quisque non ornare lorem. Nulla vulputate pulvinar libero, sit amet sodales turpis placerat id. Etiam eleifend, sem id euismod consectetur, libero leo pretium nisl, et venenatis quam est quis est. Aenean sit amet sem vel nisi finibus maximus et vitae nunc. Cras sit amet velit nec urna scelerisque porta.

\subsubsection{Subsubsection}

Vestibulum erat felis, cursus vitae convallis ac, commodo eu nisi. Nulla facilisi. Mauris dignissim nisi felis, a mollis ex accumsan vel. Suspendisse bibendum vitae nibh in suscipit. Vestibulum et finibus eros. Nulla facilisi. Cras luctus aliquam finibus. In nec justo nec orci malesuada faucibus. Nulla ultrices elit justo, vitae tristique massa vulputate sit amet.

Curabitur id placerat orci. Vivamus pulvinar augue ac feugiat blandit. Donec in ultricies mi. Nam eu lacus ac augue aliquet consectetur. Praesent dui risus, sollicitudin nec felis ut, posuere ultricies dolor. Sed massa nulla, dignissim eget sem sit amet, eleifend fermentum dui. Morbi eu feugiat velit, et tempus augue. Proin ac mattis tortor. Donec tincidunt, ante rhoncus luctus semper, arcu lorem lobortis justo, nec convallis ante quam quis lectus.

%------------------------------------------------

\section{Table Examples}

Referencing a table using its label: Table \ref{tab:onecolumn}.

\begin{table}[h] % [h] forces the table to be output where it is defined in the code (it suppresses floating)
	\centering % Centre the table
	\caption{Sausage nutrition (one column).}
	\begin{tabular}{l l l}
		\toprule
		\textit{Per 50g} & \textbf{Pork} & \textbf{Soy} \\
		\midrule
		Energy & 760kJ & 538kJ\\
		Protein & 7.0g & 9.3g\\
		Carbohydrate & 0.0g & 4.9g\\
		Fat & 16.8g & 9.1g\\
		Sodium & 0.4g & 0.4g\\
		Fibre & 0.0g & 1.4g\\
		\bottomrule
	\end{tabular}
	\label{tab:onecolumn} % For referencing the table number automatically in text with \ref{<label>}
\end{table}

Aenean feugiat pellentesque venenatis. Sed faucibus tristique tortor vel ultrices. Donec consequat tellus sapien. Nam bibendum urna mauris, eget sagittis justo gravida vel. Mauris nisi lacus, malesuada sit amet neque ut, venenatis tempor orci. Curabitur feugiat sagittis molestie. Duis euismod arcu vitae quam scelerisque facilisis. Praesent volutpat eleifend tortor, in malesuada dui egestas id. Donec finibus ac risus sed pellentesque. Donec malesuada non magna nec feugiat. Mauris eget nibh nec orci congue porttitor vitae eu erat. Sed commodo ipsum ipsum, in elementum neque gravida euismod. Cras mi lacus, pulvinar ut sapien ut, rutrum dui.

\begin{table*}[bp] % The table* environment is used for two column tables, [bp] forces the table to be output at the bottom of the page
	\centering % Centre the table
	\caption{Sausage nutrition (two columns).}
	\begin{tabular}{p{0.25\linewidth} p{0.1\linewidth} p{0.1\linewidth}}
		\toprule
		\textit{Per 50g} & \textbf{Pork} & \textbf{Soy} \\
		\midrule
		Energy & 760kJ & 538kJ\\
		Protein & 7.0g & 9.3g\\
		Carbohydrate & 0.0g & 4.9g\\
		Fat & 16.8g & 9.1g\\
		Sodium & 0.4g & 0.4g\\
		Fibre & 0.0g & 1.4g\\
		\bottomrule
	\end{tabular}
	\label{tab:twocolumn} % For referencing the table number automatically in text with \ref{<label>}
\end{table*}

Nullam mollis tellus lorem, sed congue ipsum euismod a. Donec pulvinar neque sed ligula ornare sodales. Nulla sagittis vel lectus nec laoreet. Nulla volutpat malesuada turpis at ultricies. Ut luctus velit odio, sagittis volutpat erat aliquet vel. Donec ac neque eget neque volutpat mollis. Vestibulum viverra ligula et sapien bibendum, vel vulputate ex euismod. Curabitur nec velit velit. Aliquam vulputate lorem elit, id tempus nisl finibus sit amet. Curabitur ex turpis, consequat at lectus id, imperdiet molestie augue. Curabitur eu eros molestie purus commodo hendrerit. Quisque auctor ipsum nec mauris malesuada, non fringilla nibh viverra. Quisque gravida, metus quis semper pulvinar, dolor nisl suscipit leo, vestibulum volutpat ante justo ultrices diam. Sed id facilisis turpis, et aliquet eros.

Integer velit orci, rhoncus rutrum congue at, egestas nec ex. Sed dignissim non orci vel pharetra. Aliquam nec urna ultricies, vestibulum orci a, pretium dolor. Etiam commodo iaculis condimentum. Etiam felis nulla, fringilla non ante sed, placerat tempus neque. Pellentesque porttitor lacinia risus, vitae lacinia mi fermentum in. Lorem ipsum dolor sit amet, consectetur adipiscing elit. Vivamus ante quam, semper vitae convallis ac, varius et odio. Nulla elementum sem ac congue blandit.

Duis venenatis eget lectus a aliquet. Integer vulputate ante suscipit felis feugiat rutrum. Aliquam eget dolor eu augue elementum ornare. Nulla fringilla interdum volutpat. Sed tincidunt, neque quis imperdiet hendrerit, turpis sapien ornare justo, ac blandit felis sem quis diam. Proin luctus urna sit amet felis tincidunt, sed congue nunc pellentesque. Ut faucibus a magna faucibus finibus. Etiam id mi euismod, auctor nisi eget, pretium metus. Proin tincidunt interdum mi non interdum. Donec semper luctus dolor at elementum. Aenean eu congue tortor, sed hendrerit magna. Quisque a dolor ante. Mauris semper id urna id gravida. Vestibulum mi tortor, finibus eu felis in, vehicula aliquam mi.

%------------------------------------------------

\section{Figure Examples}

Referencing a figure using its label: Figure \ref{fig:onecolumn}.

\begin{figure}[h] % [h] to output the figure where it is used in text, as opposed to letting if float 
	\centering % Centre the figure
	\includegraphics[width=0.9\linewidth]{placeholder.jpg}
	\caption{Caption for a (single-column) figure. Note that MC uses by default a single column layout.}
	\label{fig:onecolumn} % For referencing the figure number automatically in text with \ref{<label>}
\end{figure}

Integer velit orci, rhoncus rutrum congue at, egestas nec ex. Sed dignissim non orci vel pharetra. Aliquam nec urna ultricies, vestibulum orci a, pretium dolor. Etiam commodo iaculis condimentum. Etiam felis nulla, fringilla non ante sed, placerat tempus neque. Pellentesque porttitor lacinia risus, vitae lacinia mi fermentum.

Nullam ac quam sit amet eros tincidunt tristique. Donec consectetur tellus ut consectetur gravida. Vestibulum arcu lorem, consectetur sed sem in, ultricies viverra justo. Phasellus fermentum vulputate libero, eu finibus nisi. Fusce porta urna a leo faucibus, nec posuere leo tincidunt. Aenean tempor interdum lorem, in pulvinar enim fermentum at. Sed at iaculis sem. Suspendisse at diam tellus.

\begin{figure*} % The figure* environment is used for two column figures
	\centering % Centre the figure
	\includegraphics[width=0.9\linewidth]{placeholder.jpg}
	\caption{Caption for a (two-column) figure. Note that MC uses by default a single column layout.}
	\label{fig:twocolumn} % For referencing the figure number automatically in text with \ref{<label>}
\end{figure*}

%------------------------------------------------

\section{List Examples}

\subsection{Bullet Point List}

\begin{itemize}
	\item First bullet point item
	\begin{itemize}
		\item First indented bullet point item
		\item Second indented bullet point item
		\begin{itemize}
			\item First second-level indented bullet point item
		\end{itemize}
	\end{itemize}
	\item Second bullet point item
	\item Third bullet point item
\end{itemize}

\subsection{Numbered List}

\begin{enumerate}
	\item First numbered item
	\begin{enumerate}
		\item First indented numbered item
		\item Second indented numbered item
		\begin{enumerate}
			\item First second-level indented numbered item
		\end{enumerate}
	\end{enumerate}
	\item Second numbered item
	\item Third numbered item
\end{enumerate}

%------------------------------------------------

\section{Referencing Citations}

This statement requires citation \cite{Smith:2012qr}. This statement requires multiple citations \cite{Smith:2013jd, Smith:2012qr}. This statement contains an in-text citation: \textcite{Smith:2013jd}.

%------------------------------------------------

\section{Equation}

Referencing an equation using its label: \ref{eq:example}.

\begin{equation}
	\cos^3 \theta =\frac{1}{4}\cos\theta+\frac{3}{4}\cos 3\theta
	\label{eq:example}
\end{equation}

%------------------------------------------------

\section{Theorem}

\begin{theorem}[Pythagoras] 

Suppose $a\leq b\leq c$ are the side-lengths of a right triangle.\\  Then $a^2+b^2=c^2$.

\end{theorem}
%------------------------------------------------
\begin{proof}
A simple proof:
\[
a^2 + b^2 = c^2 \qedhere
\]
\end{proof}

%----------------------------------------------------------------------------------------
%	 REFERENCES
%----------------------------------------------------------------------------------------

%\printbibliography % Output the bibliography when using a .bib file

\begin{thebibliography}{}
    \bibitem{ABC00}
        A. Miller. \emph{A paper}, 2000, pp. 1--40.\\[-20pt]
    \bibitem{DEFG12}
        B. Smith et al. \emph{Amazing book}, Famous Publisher, 1912.\\[-20pt]
\end{thebibliography}

%----------------------------------------------------------------------------------------

\end{document}
